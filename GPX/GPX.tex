\documentclass{beamer}

\usepackage{amsmath}
\usepackage{graphicx}
\usepackage{framed}

\begin{document}

\begin{frame}
\Large
\begin{enumerate}
\item Create Maps With \texttt{R} Geospatial Classes and Graphics Tools (\textit{Making Maps})
\item Read and write ESRI Shape Files (\textit{ESRI})
\item Display T Spatial Objects with Google Maps and Google Earth (\textit{KML})
\item Read and Display Data from GPS Devices Using \texttt{R} (\textit{GPX})
\item Overlay Points on Satellite Image / Extract Pixel Values (\textit{Raster})
\end{enumerate}
\end{frame}

%---------------------------%
\begin{frame}
\huge
Read and Display Data from GPS Devices Using \texttt{R}
\end{frame}
%--------------------------- %
\begin{frame}
\frametitle{readGPS function}

\begin{itemize}
\item This Use Case demonstrates use of the readGPS function, included in the R maptools package. readGPS() is an R interface to the GPSBabel utility, which provides file conversion and data manipulation tools for many popular GPS data formats.

\item For this demonstration, we selected the GPS Exchange (GPX) format, GPS platform-independent and widely-used by Web-aware programs to exchange geospatial data.
\end{itemize}
\end{frame}


%--------------------------- %
\begin{frame}[fragile]
The Demonstration:
The demonstration has three parts:

Read the input files into R Data Frames
Plot the data points
Write a subset of the input data to CSV output files
Let's look at the R script, section by section. This example code contains two functions:

TheDriver(), which manages execution and generates data plots, and ConvertGPXFiles(), which extracts and filters GPS data and / writes .CSV files.


 Let's look at the R source code.

First, here is the program documentation: This describes the two functions in the package.

\end{frame}
%--------------------------- %
\begin{frame}
\frametitle{Demonstration:}
Reading and converting GPS datasets using the R readGPS function (maptools package)

\begin{itemize}
\item This demonstration reads GPX-format files containing, respectively, waypoints and tracks collected with a GPS reciever, into  an R Data Frame. 
\item Then, it extracts selected columns from
each incoming data frame (date/time/latitude/longitude/altitude)
into a new and separate data frame. 
\item The new data frame
is then written to a comma-separated-value (CSV) file.
\end{itemize}
\end{frame}
%--------------------------- %
\begin{frame}[fragile]
\frametitle{Demonstration:}
Functions:
\begin{itemize}
\item TheDriver() - Manages execution of demonstration function 
\item ConvertGPXFiles() - Reads input .GPX- format file, writes
a subset to an output .CSV file.
\end{itemize}
\end{frame}
%--------------------------- %
\begin{frame}[fragile]
Note: 
\begin{itemize} 
\item debug() and browser() statements are incorporated to allow the 
analysis to 'single-step' through the example:
\item debug() : Sets up single-line execution of the function.
\item browser(): Interrupts execution of function, allows inspection of R variable values.
\end{itemize}
\end{frame}
%--------------------------- %
\begin{frame}[fragile]
Here is TheDriver()

\begin{framed}
\begin{verbatim}
TheDriver <-function()
{
 library(maptools)
#
# debug(): single-step execution begins here.
#
 debug(ConvertGPXFiles)
 setwd("C:/Projects/scicomp/NewSite/UseCases/ReadGPXFormatGPSData")
\end{verbatim}
\end{framed}
\end{frame}
%--------------------------- %
\begin{frame}[fragile]
 convert a file containing Waypoints
\begin{verbatim}
 ConvertGPXFiles("SampleWaypointsOnlyNoOutliers.gpx","SampleWaypointsOut.csv","w")
#
# Plot the waypoints 
# 
 plot(dfWayptForPlotting@coords[,1:2],type="p",xlab="longitude",ylab="latitude")
 title("Waypoints from GPX-format GPS file")
 \end{verbatim}
Next, call ConvertGPXFiles() twice for two different input files: Once to extract waypoints(single locations), once to extract tracks (linear featues defined by a series of points):

\end{frame}

%--------------------------- %
\begin{frame}[fragile]
Convert a file containing Tracks
\begin{verbatim}
 ConvertGPXFiles("SampleTracksOnly.gpx","SampleTracksOut.csv","t")
\end{verbatim}
Finally, create a plot using the tracks data, then add ('overlay') the waypoints:
\end{frame}

%--------------------------- %
\begin{frame}[fragile]

 Plot the track. Note: with the default form (below), plot axes do not display. 
 To plot x and y axes, 'deconstruct' the x and y coordinates:
 'embedded' in the SpatialLinesDataFrame. Thus the syntax:
\end{frame}

%--------------------------- %
\begin{frame}[fragile] 
\begin{framed}
\begin{verbatim}
 print(sprintf("Plot tracks/waypoints together..."))
#browser()
 plot(sldfTracksForPlotting@lines[[1]]@Lines[[1]]@coords[,1],
 sldfTracksForPlotting@lines[[1]]@Lines[[1]]@coords[,2],
 type="l",xlab="longitude",ylab="latitude",add=TRUE)
 points(dfWayptForPlotting@coords[,1:2],type="p",col="red",pch=19) 
 title("GPX/GPS Tracks (black) | Waypoints (red)") 
# 
 print(sprintf("done with program"))
# browser()
}
\end{verbatim}
\end{framed}
\end{frame}

%--------------------------- %
\begin{frame}[fragile]
Here is ConvertGPXFiles():


Routine ConvertGPXFiles demonstrates 1) extraction of GPS data 
stored in the standard GPX format into an R data frame,
using the R readGPS function; 2) extraction of key fields
(Date/Time, Latitude/Longitude, Elevation) from the initial 
data frame into a new data frame.
\end{frame}

%--------------------------- %
\begin{frame}[fragile]
\begin{verbatim}
# function arguments: 
# 
# inGPSFile (string): Input GPX file name
# outConvertFile (string): Output CSV file name
# FileType (string): Flag indicates input
# file data type: "w" for waypoints,
# "t" for tracks.
#
\end{verbatim}
\end{frame}

%--------------------------- %
\begin{frame}[fragile]
\begin{verbatim}
ConvertGPXFiles <- function(inGPSFile,outConvertFile,FileType) 
{
# browser()
Two processing sections,based on file type: Here is the section for waypoints:

 (Consult inline comments for details)


 if (FileType == "w")
 { 
 ..
\end{verbatim} 
 
\end{frame}
%--------------------------- %
\begin{frame}[fragile]

Read the GPX-format waypoints into a Data Frame
Note here: readGPS converts waypoints and tracks into 
data frames with different column layouts. 
Columns are labeled V1 - Vn)

\begin{verbatim}
gRawWaypt = readGPS("gpx",inGPSFile,"w")
\end{verbatim}
\end{frame}

%--------------------------- %
\begin{frame}[fragile]

Get number of observations (waypoints)
\texttt{gRawWaypoint} data frame contains the attributes that we desire in the following columns:
\begin{itemize}
\item V3: Observation Date (factor)
\item V4: Observation Time (factor)
\item V8: Descruptive Label (string)
\item V10: Latitude (numeric)
\item V11: Longitude (numeric)
\item V21: Elevation (Factor, includes M in last character position)
\end{itemize}

\end{frame}

%--------------------------- %
\begin{frame}
\begin{itemize}
\item Lets extract these columns, convert Date, Time to strings 
\item and elevation to numeric format, and construct a new data frame
\item containing only the columns of interest.
\end{itemize}
\end{frame}
%--------------------------- %
\begin{frame}[fragile]
\begin{framed}
\begin{verbatim}
 nObs = length(gRawWaypt[,"V3"])
 sDate = as.character(gRawWaypt[,"V3"])
 sTime = as.character(gRawWaypt[,"V4"])
 sLabel = as.character(gRawWaypt[,"V9"]) 
 fLat = as.numeric(gRawWaypt[,"V10"])
 fLong = as.numeric(gRawWaypt[,"V11"])
#
# compute the spatial bounding box for the waypoint set:
#
 bounds = c(range(fLong),range(fLat))
 dim(bounds) = c(2,2)
\end{verbatim}
\end{framed}
\end{frame}
%--------------------------- %
\begin{frame}[fragile] 
Elevation is a factor with the letter 'M' appended.
Remove this, and convert the elevation to numeric
This formula extracted from the HTML help file for the R readGPS package

\begin{verbatim}
fAlt <- as.numeric(substring(as.character(gRawWaypt$V21),
 1,(nchar(as.character(gRawWaypt$V21))-1)))
\end{verbatim}
Output data frames - One 'standard' DF for file output,
one 'SpatialPointsDataFrame' for plotting.
\end{frame}

 %--------------------------- %
\begin{frame}[fragile]
\begin{verbatim}
 dfWaypoints <<- as.data.frame(cbind(sLabel,sDate,sTime,fLat,fLong,fAlt)) 
 write.csv(dfWaypoints,outConvertFile)
 LatLongCoords = SpatialPoints(cbind(fLong,fLat),proj4string = CRS("+proj=longlat"))
 dfWayptForPlotting <<- SpatialPointsDataFrame(LatLongCoords,
 bbox=as.matrix(bounds),dfWaypoints[1])
 print(sprintf("done - waypoints"))
 }
\end{verbatim}
\end{frame}

 %--------------------------- %
\begin{frame}[fragile]
Here is the section for tracks:

\begin{verbatim}
 
 else if (FileType == "t")
 { 
 gRawTracks = readGPS("gpx",inGPSFile,"t")

\end{verbatim}
A GPX file can include multiple tracks, known as track sequences, each of which contains the vertices for a single track (line).
However, the current version of \texttt{readGPS()} combines all points in all  track sequences into a single track. (This may change in the future).

\end{frame}


%--------------------------- %
\begin{frame}
\begin{itemize}
\item For this demonstration, we will convert these track points into a 
 SpatialLinesData Frame (with a single SpatialLines object in one 'row'
 and a single attribute - "Track 1" - attached to the track (and stored
in a DataFrame).
\item The data frame generated by readGPS for Tracks has a different layout:

 V3: Latitude (numeric) 
 V4: Longitude (numeric)
 V14: Elevation (Factor, includes 'M' in last character position)
\end{itemize} 
\end{frame}

%--------------------------- %
\begin{frame}
We use two \texttt{R} geospatial data structures, provided by the sppackage:
\begin{itemize}
\item \texttt{SpatialPointsDataFrame} for waypoints, \item \texttt{SpatialLinesDataFrame} for tracks.
\end{itemize}
 \textit{Consult the sp package documentation for details on spatial Data Frame components.
}
\end{frame}

%--------------------------- %
\begin{frame}[fragile]
SpatialPointsDataFrame: 
One attribute (altitude) per point.

\begin{framed}
\begin{verbatim} 
 dfTrackPoints <<- SpatialPointsDataFrame(LatLongCoords,data.frame(fAlt))
\end{verbatim}
\end{framed}
\end{frame}
%--------------------------- %
\begin{frame}[fragile]
Create SpatialLinesDF from the SpatialPointsDF.
 First, SpatialLines object:

\begin{framed}
\begin{verbatim}
 slTrackLine = Lines(list(Line(dfTrackPoints@coords)))
 slTrackLine@ID = "Track_One"
 SLPath = SpatialLines(list(slTrackLine),proj4string = CRS("+proj=longlat"))

\end{verbatim}
\end{framed}
\end{frame}

%--------------------------- %
\begin{frame}

Finally, the \texttt{SpatialLinesDataFrame}:

\begin{itemize}
\item Create a global variable using the "<<-" operator
\item Plot this in TheDriver().
\item Write the original track points to a CSV file.
\end{itemize}
\end{frame}

%--------------------------- %
\begin{frame}[fragile]
\begin{framed}
\begin{verbatim}
 sldfTracksForPlotting <<- SpatialLinesDataFrame(SLPath,TrackAttributes,match.ID=FALSE)

 write.csv(dfTrackPoints,outConvertFile)
 print(sprintf("done - tracks")) 
 }
}
\end{verbatim}
\end{framed}
\end{frame}

%--------------------------- %
\begin{frame}[fragile]
Here is the \texttt{R} command sequence that loads and runs the program:

\begin{framed}
\begin{verbatim}
> source("ExtractWaypoints.r")
> TheDriver()
\end{verbatim}
\end{framed}
Here are the plots. Note that two 'outlier' waypoints are omitted from the right-side plot as they do not fall within the tracks.
\end{frame}
%--------------------------- %


\end{document}
