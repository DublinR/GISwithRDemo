Flood Risk Management

Flooding is a natural process that can happen at any time in a wide variety of locations. Flooding from the sea and rivers is probably best known but prolonged, intense and localised rainfall can also cause sewer flooding, overland flow and groundwater flooding. Flooding has significant impacts on human activities. 
It can threaten people’s lives and their property, and inaddition to economic and social damage, floods can have severe environmental consequences.

 

There is therefore a need to manage and minimise future flood risk. Land use management and spatial planning has a key role to play with respect to flood risk management, in particular in ensuring that future development avoids or minimise increases in flood risk.

 

The Planning System and Flood Risk Management-Guidelines for Planning Authorities (DEHLG and OPW, 2009) were issued by the Minister of the Environment, Heritage and Local Government under Section 28 of the Planning and Development Act (2000) as amended. Planning authorities and An Bord Pleanála are required to have regard to the guidelines when carrying out their functions under the Planning Acts. 
