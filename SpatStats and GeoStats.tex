\subsection{Spatial Statistics}
\begin{verbatim}
### SPATIAL STATISTICS ###
 
## Point pattern analysis
  library(spatial)
  library(spatstat)
  library(spatgraphs)
  library(ecespa)    # ecological focus
  # etc (see Spatial Task View)
 
  # example
  data(fig1)
  plot(fig1)    # point pattern
  data(Helianthemum)
  cosa12 <- K1K2(Helianthemum, j="deadpl", i="survpl", r=seq(0,200,le=201),
         nsim=99, nrank=1, correction="isotropic")
  plot(cosa12$k1k2, lty=c(2, 1, 2), col=c(2, 1, 2), xlim=c(0, 200),
         main= "survival- death",ylab=expression(K[1]-K[2]), legend=FALSE)
 
\end{verbatim}
\subsection{Geostatistics}
The ackage \textbf{\textit{gstat}} provides a wide range of uni-ariable and multivariable geostatistical modelling, prediction and simulation functions.

\begin{verbatim}
### Geostatistics ###
  library(gstat)
  library(geoR)
  library(akima)   # for spline interpolation
  # etc (see Spatial Task View)
 
 
  library(spdep)   # dealing with spatial dependence
 
 \end{verbatim}
When analyzing geospatial data, describing the spatial pattern of a measured variable is of great importance.  A common way of visualizing the spatial autocorrelation of a variable is a variogram plot
